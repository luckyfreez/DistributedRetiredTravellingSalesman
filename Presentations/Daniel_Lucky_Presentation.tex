\documentclass[10pt, xcolor=svgnames]{beamer}
\mode<presentation>{\usetheme[compress]{Berlin}}
\setbeamertemplate{theorems}[ams style]
\setbeamertemplate{caption}[numbered]
\usepackage{amsmath,amssymb,latexsym,graphicx,verbatim}

% CUSTOM COMMANDS
\newtheorem{Conjecture}{Conjecture}
\newtheorem{Question}{Question}
\newtheorem{Observation}{Observation}
\newcommand{\del}{\mathrm{\Delta}}
\newcommand\Fontvi{\fontsize{6}{7.2}\selectfont}
\DeclareMathOperator*{\argmin}{arg\,min}
\DeclareMathOperator*{\argmax}{arg\,max}


% THE PRESENTATION
\title[Distributed TSP]{The Distributed Retired Traveling Salesman}
\subtitle{Applying Integer Programming Concepts to Solve a Harder Version of the Traveling Salesman Problem}
\author[Seita and Zhang]{Daniel Seita and Lucky Zhang} 
\institute[Williams College]{Department of Computer Science, Williams College}
\date{\today}

\begin{document}
\begin{frame}
	\titlepage
\end{frame}

%\begin{figure}[p]
%\centering
%\includegraphics[width=0.5\textwidth]{IMAGES}
%\caption{}
%\label{fig:...}
%\end{figure}


\section{``Retired'' TSP}

\subsection{Traveling Salesman Problem}

\begin{frame}
\begin{Problem}[Traveling Salesman Problem]
Given a list of cities and their pairwise distances, what is the shortest possible route that visits each city exactly once and returns to the
starting city?
\end{Problem}

\begin{itemize}
    \item One of the most studied problems in all of computer science.
    \item Decision version (does there exist such a cycle with cost $\le B$) is NP-complete\footnote{Recall from CS 256: NP-complete means (1) in NP
and (2) NP-hard.}.
\end{itemize}
\end{frame}

\begin{frame}
Our problem is harder!
\begin{Problem}[Distributed Retired TSP]
We have the following (real-world) problem:
\begin{itemize}
    \item A list of cities $a_1, a_2, \ldots, a_n$
    \item A starting date $d_{start}$ and ending date $d_{end}$.
\end{itemize}
The goal is to find \textbf{the minimum cost} of any a series of flights $f_1, \ldots, f_m$ such that we visit every city \textbf{at least once}.
Hence, ``Retired'' TSP because you'd need to be wealthy and retired, and ``Distributed'' because we plan on using multiple machines to speed up
algorithm. 
\end{Problem}
\pause
Biggest challenge: for any two cities, costs are no longer fixed! 
\end{frame}


\section{Binary Integer Programming Formulation}

\subsection{Binary Integer Programming}

\begin{frame}
We use \textbf{binary integer programming} to help us solve Retired TSP.

\begin{itemize}
    \item $\text{minimize }   \mathbf{c}^\mathrm{T} \mathbf{x} $
    \item $\text{subject to } A \mathbf{x} \ge \mathbf{b},$
    \item $\text{and } x_i \in \{0,1\}$
\end{itemize}

\pause
Here is how we \textbf{formulated} Retired TSP as a binary IP problem:

\begin{itemize}
    \item Let $c_{ijt}$ denote the \emph{minimum cost} of flying from city $i$ to $j$ on day $t$.
\end{itemize}
Our key variables:
\[
x_{ijt} = \begin{cases}
1 &\mbox{if we go from cities } i \mbox{ to } j \mbox{ on day } t, \\ 
0 & \mbox{otherwise}.
\end{cases}
\]
\end{frame}

\begin{frame}
Here are our first three constraints:

\begin{equation}
\sum_{i=1}^{n} \sum_{j=1}^{n} x_{ijt} \le 1 \mbox{ for all } t \in \{1, 2, \ldots, m\}.
\end{equation}

Ensures we have at most one flight per day.

\pause
\begin{equation}
\sum_{t=1}^{m} \sum_{i=1}^{n} x_{ijt} \ge 1 \mbox{ for all } j \in \{1, 2, \ldots, n\}.
\end{equation}

Ensures we enter each city at most once.

\pause
\begin{equation}
\sum_{t=1}^{m} \sum_{j=1}^{n} x_{ijt} \ge 1 \mbox{ for all } i \in \{1, 2, \ldots, n\}.
\end{equation}

Ensures we leave each city at most once.\\
\pause

\textbf{What's the problem?}
\end{frame}

\begin{frame}
Problem 1: Doesn't prevent cycles!
\begin{itemize}
    \item Flight 1: BOS to NYC
    \item Flight 2: NYC to BOS 
    \item Flight 3: SEA to SFO 
    \item Flight 4: SFO to SEA
\end{itemize}
\pause
Problem 2: Doesn't prevent logistical impossibilities!
\begin{itemize}
    \item Flight 1 on June 2: BOS to NYC
    \item Flight 2 on June 5: SEA to BOS
    \item Flight 3 on June 8: NYC to SEA
\end{itemize}
\end{frame}


\section{Implementation and Results}

\subsection{Implementation}

\begin{frame}
Our implementation:

\begin{itemize}
    \item We have a master server that sets up and calls everything (uses \textbf{xmlrpc}!)
    \item We have a web crawler that crawls \textbf{Matrix Airfare Search} to obtain prices
    \item We have a slave server that gets called by the master, and then uses the web crawler to obtain prices.
    \item The algorithm to solve binary IP is \textbf{Balas' Additive Algorithm}. Basic idea: depth-first search of all possible variable assignments
with pruning and look-aheads to avoid searching ``impossible'' solutions.
\end{itemize}
Let's do a 3 city, 3 day demonstration!
\end{frame}

\subsection{Results and Thoughts}

\begin{frame}
\begin{itemize}
    \item We ran it on an example with 4 cities over the span of 6 possible travel days. This means $4 \cdot 3 \cdot 6 = 72$ total variables $X_{ijt}$.
    \item Our solution worked and it analyzed 73,976 nodes in the DFS tree, rather than the full $2^{72}$ set of solutions.
    \item Took just about 10 minutes, and price-checking took almost all the time.
    \item This is still an exponential time solution in general, but \textbf{much} faster than brute force.
\end{itemize}
\end{frame}

\begin{frame}
Still a number of things to do before May 23:
\begin{itemize}
    \item \textbf{Distribute price checking} (e.g., using PlanetLab)
    \item Evaluate speed and practical usefulness
    \item Expand to get multiple flights for one ticket
    \item Make a front-end server for the ignorant masses
    \item Finish paper: 8 pages overview, 15 pages of dense, technical details
\end{itemize}
\end{frame}




\section{}
\begin{frame}
\begin{center}
\Large{Thank you for your attention.}

~

\Huge{Any questions?\\}
\end{center}

\begin{center}
Please contact one of us if you have any comments or feedback about our presentation. We view it as a privilege to be able to give these talks, and
want to make sure that our viewers are as happy as possible.
\end{center}
\end{frame}

\end{document}
